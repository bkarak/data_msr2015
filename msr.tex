\pdfpagewidth=8.5in 
\pdfpageheight=11in


\documentclass{sig-alternate}
\usepackage[utf8]{inputenc} 
\usepackage[T1]{fontenc}
\usepackage{microtype}

\usepackage{url}
\usepackage{amsmath}
\usepackage{graphicx}
\usepackage{subfigure}
\usepackage{threeparttable}
\usepackage{pdflscape}
\usepackage{array}
\usepackage{color}

\usepackage{flushend}

\clubpenalty = 10000
\widowpenalty = 10000
\displaywidowpenalty = 10000

\begin{document}

\conferenceinfo{MSR}{'15, May 16 – 17, 2015, Florence, Italy}
\CopyrightYear{2015}
\crdata{978-1-4503-2863-0/14/05}

\title{The Three Corridors}

\numberofauthors{5}

\def\aueb{\textsuperscript{*}}
\def\tud{\textsuperscript{\dag}}

\author{
  Vassilios Karakoidas\aueb \and Dimitris Mitropoulos\aueb \and Panos Louridas\aueb \and Georgios Gousios\tud \and Diomidis Spinellis\aueb \\
  \begin{tabular}{ccc}
  \affaddr{\aueb Department of Management Science and Technology} & & \affaddr{\tud Software Engineering Research Group}\\
    \affaddr{\aueb Athens University of Economics and Business} & & \affaddr{\tud Delft University of Technology}\\
   \affaddr{Athens, Greece} & & \affaddr{Delft, the Netherlands}\\
   \email{\{bkarak,dimitro,louridas,dds\}@aueb.gr}& & \email{g.gousios@tudelft.nl} \\
  \end{tabular}
}

\maketitle
\begin{abstract}

\end{abstract}

\category{D.2.4}{Software Engineering}{Software/Program Verification}[Statistical methods]
\category{D.2.7}{Software Engineering}{Distribution, Maintenance, and Enhancement}[Version control]

\terms{Static Analysis, Software Ecosystems.}

\keywords{Maven Repository, JDepend, Software Bugs.}


\section{Introduction}
\label{sec:intro}
 

\section{Dataset Contents}
\label{sec:data}

\begin{table}
\centering
\caption{The selected Maven projects' size metrics}
\label{tbl:oss-size-metrics}
\begin{tabular}{l r}
 \hline
\textbf{Metric} & \textbf{Value}\\ 
\hline
Project Count & 12,959\\
File Count & 604,821\\
Module Count & 72,302\\
Lines of Code & 110,156,703\\
Source Lines of Code & 61,246,807\\
Comment Lines of Code & 36,696,217\\
Number of Classes & 499,588\\
Number of Interfaces & 89,145\\
Number of Enumerations & 12,732\\
\hline
\end{tabular}
\end{table}

\begin{table}
\centering
\caption{Selected Metrics}
\label{tbl:selected-metrics}
\begin{tabular}{l l}
 \hline
\multicolumn{2}{l}{\textit{\textbf{Class \& Method Design}}}\\
\hline
Depth Of Inheritance Tree & \textit{ckjm}\\
Coupling Between Objects & \textit{ckjm}\\
Weighted Methods Per Class & \textit{ckjm}\\
Response For Class & \textit{ckjm}\\
Lack Of Cohesion In Methods & \textit{ckjm}\\
Number Of Children & \textit{ckjm}\\
Attribute Hiding Factor & \textit{clmt}\\
Coupling Between Methods & \textit{ckjm}\\ 
Average Method Complexity & \textit{ckjm}\\
Cohesion Among Methods of Class & \textit{ckjm}\\
Data Access Metric & \textit{ckjm}\\ 
Inheritance Coupling & \textit{ckjm}\\
Lack Of Cohesion In Methods3 & \textit{ckjm}\\ 
McCabe Cyclomatic Complexity & \textit{ckjm}\\
Measure Of Aggregation & \textit{ckjm}\\
Measure Of Functional Abstraction & \textit{ckjm}\\
Method Hiding Factor & \textit{clmt}\\
\hline
\multicolumn{2}{l}{\textit{\textbf{Package Design}}}\\
\hline
Afferent Couplings & \textit{jdep}\\
Efferent Couplings & \textit{jdep}\\
Instability & \textit{jdep}\\
Abstractness & \textit{jdep}\\
Distance Main Sequence & \textit{jdep}\\
\hline
\multicolumn{2}{l}{\textit{\textbf{Program Size Metrics}}}\\
\hline
Comments Lines Of Code & \textit{clmt}\\
Lines Of Code & \textit{clmt}\\
Native Methods Per Code Unit & \textit{clmt}\\
Native Methods Per Project & \textit{clmt}\\
Source Lines Of Code & \textit{clmt}\\ 
Function Oriented Code & \textit{clmt}\\
\hline
\end{tabular}
\end{table}

\section{Harnessing Our Dataset}
\label{sec:exploit}


\section{Limitations}
\label{sec:limit}


\section{On the Use of Domain Specific Languages}
\label{sec:dsl}

The Java Platform exists more than a decade and provides a development environment for several application areas, such as web, grid computing, and traditional client-server applications. The official development environment of the Java platform is released by Oracle and is widely-known as the Java Standard Edition SDK. The process of development and innovation in the Java ecosystem is built around a community, that writes proposals of features, known as {\sc jsr}s (Java Specification Requests), which are supported by prototype implementations.

In the literature, it is referred that {\sc dsl}s are used to reduce software development cost, by introducing domain efficiency \cite{MHS05}. This thesis focuses on practical research aspects as well as with the theoretical ones; thus the first problem that needs to be examined, is to measure the adoption of {\sc dsl}s in software projects. In addition, the average number of {\sc dsl}s usually used in a software project needs to be investigated.

The methodology was the following; A set of standard {\sc dsl} application libraries was identified, and the source was scanned for specific \textit{import} statements e.g. \textit{java.util.regex}, which indicated that the standard package that implements regular expressions was used, thus regular expressions were used. If a package is detected in the source code, then the project will be tagged that it uses one (1) {\sc dsl}. Consequently, if a project has {\sc dsl} count four (4), then it means that four (4) different application libraries were detected during the source code scan. Table \ref{tbl:dsl-list} lists the selected {\sc dsl}s application libraries. Note that all these libraries are included as part of the Java {\sc sdk}.

\begin{table}
\centering
\caption{List of selected DSL application libraries}
\label{tbl:dsl-list}
\begin{tabular}{l l}
 \hline
\textbf{DSL} & \textbf{Java Package}\\
\hline
Regular Expressions & \verb|java.util.regex|\\
XML & \verb|javax.xml|, \verb|org.w3c| and \verb|org.xml|\\
SQL & \verb|java.sql| and \verb|javax.sql|\\
XPath & \verb|java.xml.xpath|\\
XSLT & \verb|javax.xml.transform|\\
RTF & \verb|javax.swing.text.rtf|\\
HTML & \verb|javax.swing.text.html|\\
\hline
\end{tabular}
\end{table}

The initial goal of this experiment was simply to provide quantifiable results that are indicative regarding the usage of each {\sc dsl}, thus only files containing Java code were taken into account. Build files or other resources that may contain {\sc dsl}s were not included.

One final assumption was also made; if {\sc xpath} or {\sc xslt} were found in the source code, then the project would be marked that it also uses {\sc xml}, since those two languages are used for query and transformations on {\sc xml} {\sc dom} trees.

\section{Related Work}
\label{sec:rel}

The Maven ecosystem has been previously analyzed by
Raemaekers et al.~\cite{RDV13}
to produce the {\it Maven dependency dataset}.
Apart from basic information like individual methods, classes,
packages and lines of code for every {\sc jar}, this dataset
also includes a database with all the
connections between the aforementioned elements.
Our work differs from this research because it
reports all bugs coming from the output of a
static analysis tool, for each {\sc jar}
contained in the Maven repository.

\cite{MKLGS14}

\section{Conclusions}
\label{sec:conc}


\bibliographystyle{abbrv}
\bibliography{msr}  

\end{document}
